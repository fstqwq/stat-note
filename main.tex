\documentclass[a4paper, landscape,10pt]{article}

\usepackage[inner=0.6cm, outer=0.6cm, top=0.6cm, bottom=0.65cm]{geometry}

\usepackage{xeCJK}

\setCJKmainfont{Source Han Serif SC}[Scale=0.775]
%\setCJKmonofont{SimHei}[Scale=0.5]
%\setCJKsansfont{KaiTi}[Scale=0.5]

%\setmainfont{Linux Libertine O}[Scale=0.5]
%\setmonofont{Consolas}[Scale=0.5]
%\setsansfont{Gill Sans Medium}[Scale=0.5]

\XeTeXlinebreaklocale "zh"
\XeTeXlinebreakskip = 0pt plus 1pt

\setlength{\parindent}{0em}\setlength{\parskip}{0pt}
\setlength\itemsep{0pt}

\makeatletter
\renewcommand{\paragraph}{%
  \@startsection{paragraph}{4}%
  {\z@}{0pt \@plus 1pt \@minus 1pt}{-1em}%
  {\normalfont\normalsize\bfseries}%
}
\makeatother
\usepackage[lining,semibold,type1]{libertine} % a bit lighter than Times--no osf in math
\usepackage[T1]{fontenc} % best for Western European languages

\usepackage{amsthm,amsmath,amssymb}
\usepackage[table]{xcolor}
%\usepackage[bookmarks=true,hypertexnames=false,pagebackref]{hyperref}
%\hypersetup{colorlinks=true, citecolor=blue, linkcolor=red,
%  urlcolor=blue}
\usepackage{textcomp} % required to get special symbols
\usepackage[varqu,varl]{inconsolata}% a typewriter font must be defined
\usepackage[libertine,vvarbb]{newtxmath}
\usepackage[scr=rsfso]{mathalfa}
\usepackage{bm}
\usepackage{cleveref}
\usepackage{graphicx}
\usepackage{booktabs}

\newtheorem{theorem}{Theorem}
\newtheorem{lemma}[theorem]{Lemma}
\newtheorem{corollary}[theorem]{Corollary}
\newtheorem{fact}[theorem]{Fact}
\newtheorem{definition}[theorem]{Definition}
\newtheorem{definitions}[theorem]{Definitions}
\newtheorem{conjecture}[theorem]{Conjecture}
\newtheorem{claim}[theorem]{Claim}
\newtheorem*{myclaim}{Claim}
\newtheorem*{remark}{Remark}
\newtheorem{proposition}[theorem]{Proposition}
\newtheorem{condition}{Condition}

\newcommand{\norm}[1]{\left\Vert#1\right\Vert}
\newcommand{\abs}[1]{\left\vert#1\right\vert}
\newcommand{\set}[1]{\left\{#1\right\}}
\newcommand{\tuple}[1]{\left(#1\right)} \newcommand{\eps}{\varepsilon}
\newcommand{\inner}[2]{\langle #1,#2\rangle} \newcommand{\tp}{\tuple}
\renewcommand{\mid}{\;\middle\vert\;} \newcommand{\cmid}{\,:\,}
\newcommand{\numP}{\#\mathbf{P}} \renewcommand{\P}{\mathbf{P}}
\newcommand{\defeq}{\triangleq} \renewcommand{\d}{\,\-d}
\newcommand{\ol}{\overline}

\usepackage{algorithm}
\usepackage{algorithmicx}
%\renewcommand{\thealgorithm}{} %% disable the algorithm counter
\usepackage{algpseudocode}


\def\*#1{\mathbf{#1}} \def\+#1{\mathcal{#1}} \def\-#1{\mathrm{#1}} \def\^#1{\mathbb{#1}} \def\$#1{\mathtt{#1}}


\usepackage{xifthen}
\renewcommand{\Pr}[2][]{ \ifthenelse{\isempty{#1}}
  {{P}\left(#2\right)}
  {\mathop{P}_{#1}\left(#2\right)} }
\renewcommand{\P}[2][]{ \ifthenelse{\isempty{#1}}
  {{P}\left(#2\right)}
  {\mathop{P}_{#1}\left(#2\right)} }
\newcommand{\E}[2][]{ \ifthenelse{\isempty{#1}}
  {{E}\left(#2\right)}
  {\mathop{E}_{#1}\left(#2\right)} }
\newcommand{\Var}[2][]{ \ifthenelse{\isempty{#1}}
  {{V}\left(#2\right)}
  {\mathop{{V}}_{#1}\left(#2\right)} }
  
\newcommand{\Cov}[2][]{ \ifthenelse{\isempty{#1}}
  {{Cov}\left(#2\right)}
  {\mathop{{Cov}}_{#1}\left(#2\right)} }

\usepackage{multicol}
\usepackage{titlesec}
%configure section style
\titleformat{\section}
{\large\bfseries}			% The style of the section title
{\thesection.}				% a prefix
{0pt}						% How much space exists between the prefix and the title
{}					% How the section is represented
% \titleformat{\section}{\huge}{}{0pt}{}
\titlespacing{\section}{0pt}{0pt}{0pt}
\titlespacing{\subsection}{0pt}{0pt}{0pt}
\titlespacing{\subsubsection}{0pt}{0pt}{0pt}

\begin{document}
	\small
	\begin{multicols}{2}
		\begin{tabular}{lllll}
			\multicolumn{5}{l}{{\Large{基础数理统计 2023 Spring}}} \\
			\toprule
			Discrete dist. & pmf & mean & variance & mgf/moment \\
			\midrule
			{Discrete Uniform}$(n)$ & $\frac{1}{n}$ & $\frac{n+1}{2}$ & $\frac{n^2-1}{12}$ & $\frac {e^t (1 - e^{n t}) } {n (1 - e^t) }=\frac 1n \sum e^{it}$ \\
			\rowcolor{gray!15} {Bernoulli}$(p)$ & $p^x(1-p)^{1-x}$ & $p$ & $p(1-p)$ & $(1-p) + pe^t$ \\
			{Binomial}$(n, p)$ & $\binom{n}{x}p^x(1-p)^{n-x}$ & $np$ & $np(1-p)$ & $((1-p) + pe^t)^n$ \\
			\rowcolor{gray!15}{Geometric}$(p)$ & $(1-p)^{x-1}p$ & $\frac{1}{p}$ & $\frac{1-p}{p^2}$ & $\frac{pe^t}{1-(1-p)e^t}$ \\
			{Poisson}$(\lambda)$ & $\displaystyle\frac{\lambda^x}{x!}e^{-\lambda}$ & $\lambda$ & $\lambda$ & $\displaystyle e^{\lambda(e^t-1)}$ \\
			%& \scriptsize{If $Y$ is Gamma($\alpha, \beta$), X is Poisson($\frac x  \beta$), and $\alpha$ is integer, then $P(X\geq \alpha) = P(Y\leq y)$.} &&& \\
			\rowcolor{gray!15}\shortstack{{Beta-binomial}\\$(n, \alpha, \beta)$}
			& \shortstack{$\binom{n}{x} \frac {\Gamma (\alpha + \beta)} {\Gamma (\alpha) \Gamma (\beta)} \cdot $ \\
			$\frac {\Gamma (x + \alpha) \Gamma (n - x + \beta)} {\Gamma (n + \alpha + \beta)}$}
			& $\displaystyle\frac{n \alpha}{\alpha+\beta}$
			& $\displaystyle\frac{n \alpha \beta} {(\alpha + \beta)^2}$ & 
			\scriptsize{\shortstack{If $X|P$ is Binomial$(n, P)$, \\and $P$ is Beta($\alpha, \beta$), then $X$ is \\Beta-binomial$(n, \alpha, \beta)$.}} \\
			\bottomrule
		\end{tabular}

		% Calculus Table
% \def\sinh{\mathop{\rm sinh}\nolimits}
% \def\cosh{\mathop{\rm cosh}\nolimits}
% \def\sech{\mathop{\rm sech}\nolimits}
% \def\csch{\mathop{\rm csch}\nolimits}
% \def\coth{\mathop{\rm coth}\nolimits}
% \def\tanh{\mathop{\rm tanh}\nolimits}

% \def\arccot{\mathop{\rm arccot}\nolimits}
% \def\arcsec{\mathop{\rm arcsec}\nolimits}
% \def\arccsc{\mathop{\rm arccsc}\nolimits}
% \def\arcsinh{\mathop{\rm arcsinh}\nolimits}
% \def\arccosh{\mathop{\rm arccosh}\nolimits}
% \def\arctanh{\mathop{\rm arctanh}\nolimits}
% \def\arccoth{\mathop{\rm arccoth}\nolimits}
% \def\arcsech{\mathop{\rm arcsech}\nolimits}
% \def\arccsch{\mathop{\rm arccsch}\nolimits}

\begin{multicols}{4}
	$(uv)' = uv' + u'v$ \\
	$ ({u \over v})' = {u'v - uv' \over v^2 } $ \\
	\begin{scriptsize}
	$ \int uv'\d x = uv - \int u' v \d x$ \\
	$ (f(g(x)))' = f'(g(x)) g'(x) $ \\
\end{scriptsize}
	$ (a^x)' = (\ln a) a^x $ \\
	% $ (\tan x)' = \sec^2 x $ \\
	% $ (\cot x)' = \csc^2 x $ \\
	% $ (\sec x)' = \tan x\, \sec x $ \\
	% $ (\csc x)' = - \cot x\, \csc x $ \\
	$ (\arcsin x)' = {1 \over \sqrt{1-x^2}} $ \\
	$ (\arccos x)' = -{1 \over \sqrt{1-x^2}} $ \\
	$ (\arctan x)' = {1 \over 1+x^2} $ \\
	% $ (\arccot x)' = -{1 \over 1+x^2} $ \\
	% $ (\arccsc x)' = -{1 \over x \sqrt{1-x^2}} $ \\
	% $ (\arcsec x)' = {1 \over x \sqrt{1-x^2}} $ \\
	% $ (\tanh x)' = \sech^2 x $ \\
	% $ (\coth x)' = -\csch^2 x $ \\
	% $ (\sech x)' =$ \begin{tiny}$-\sech x \, \tanh x$\end{tiny} \\
	% $ (\csch x)' =$ \begin{tiny}$-\csch x \, \coth x$\end{tiny} \\
	% $ (\arcsinh x)' = {1 \over \sqrt{1+x^2}} $ \\
	% $ (\arccosh x)' = {1 \over \sqrt{x^2-1}} $ \\
	% $ (\arctanh x)' = {1 \over 1-x^2} $ \\
	% $ (\arccoth x)' = {1 \over x^2-1} $ \\
	% $ (\arccsch x)' = -{1 \over \vert x \vert \sqrt{1+x^2}} $ \\
	% $ (\arcsech x)' = -{1 \over x \sqrt{1-x^2}} $ \\
\end{multicols}

		\begin{tabular}{lllll}
			\toprule
			Continuous dist. & pdf & mean & variance & mgf/moment \\
			\midrule
			{Uniform}$(a, b)$ & $\displaystyle\frac{1}{b-a}$ & $\displaystyle\frac{a+b}{2}$ & $\displaystyle\frac{(b-a)^2}{12}$ & $\displaystyle\frac{e^{tb}-e^{ta}}{t(b-a)}$ \\
			\rowcolor{gray!15}{Exponential}$(\theta)$ & $\displaystyle\frac{1}{\theta}e^{-\frac{x}{\theta}}$ & $\theta$ & $\theta^2$ & $\displaystyle\frac{1}{1-\theta t}, t < \frac{1}{\theta}$ \\
			\rowcolor{gray!15}{Exponential}$(\lambda)$ & $\lambda e^{-\lambda x}$ & $\displaystyle\frac{1}{\lambda}$ & $\displaystyle\frac{1}{\lambda^2}$ & $\displaystyle\frac{\lambda}{\lambda-t}, t < \lambda$ \\
			{Normal}$(\mu, \sigma^2)$ & \large{$\displaystyle\frac{1}{\sqrt{2\pi\sigma^2}}e^{-\frac{(x-\mu)^2}{2\sigma^2}}$} & $\mu$ & $\sigma^2$ & $e^{\mu t + \frac{\sigma^2 t^2}{2}}$ \\
			\rowcolor{gray!15}{Gamma}$(\alpha, \beta)$ & $\displaystyle\frac{1}{\Gamma(\alpha)\beta^\alpha}x^{\alpha-1}e^{-x / \beta}$ & $\alpha\beta$ & $\alpha\beta^2$ & $\displaystyle\left(\frac{1}{1-\beta t}\right)^\alpha, t < \frac{1}{\beta}$ \\
			{Beta}$(\alpha, \beta)$ & $\displaystyle \frac{\Gamma(\alpha+\beta)}{\Gamma(\alpha)\Gamma(\beta)}x^{\alpha-1}(1-x)^{\beta-1}$ & $\displaystyle \frac{\alpha}{\alpha+\beta}$ & $ \frac{\alpha\beta}{(\alpha+\beta)^2(\alpha+\beta+1)}$ & $1 + \sum_{k=1}^{\infty} \left(\prod_{i=0}^{k-1} \frac{\alpha+i}{\alpha+\beta+i}\right) \frac{t^k}{k!}$ \\
			\rowcolor{gray!15}{Cauchy} & $1 / \left( \pi (1 + x ^ 2)\right), x \in \mathbb R $ & ($\infty$) & n/a, \;\; $F_X(x)$ & $= \arctan(x)/\pi + 1/2$\\
			{$\chi^2_p = \sum_{i = 1}^p Z_i^2$} & $\displaystyle \frac{1}{2^{p/2}\Gamma(p/2)}x^{p/2-1}e^{-x/2}$ & $p$ & $2p$ & $(1-2t)^{-p/2}, t < 1/2$ \\
			%\rowcolor{gray!15}{Student's t}$(\nu)$ & $\frac{\Gamma((\nu+1)/2)}{\sqrt{\nu\pi}\Gamma(\nu/2)}\left(1+\frac{x^2}{\nu}\right)^{-(\nu+1)/2}$ & $0, \nu > 1$ & $ \frac \nu {\nu - 2}, \nu > 2$ & $EX^{n=2k}: \frac {\Gamma((\nu + 1) / 2) \Gamma{\nu - n / 2}} {\sqrt{\pi} \Gamma(\frac \nu 2)} {\nu^{n/2}}$\\
			\bottomrule
		\end{tabular}
	\end{multicols}
	\begin{multicols}{4}
		\section{概率}
条件概率 $P(A|B)=P(AB)/P(B)$\\
全概率 $P(B)=\sum_{i}P(B|A_i)P(A_i)$\\
贝叶斯 $P(A_i |B) = \frac{P(B|A_i)P(A_i)}{\sum_j P(B|A_j)P(A_j)}$\\
作业: 蓝眼 $1/4$. 对于三个孩子的家庭, 如果至少有一个孩子是蓝眼睛(A), 至少有两个蓝眼睛(B)的概率为
$P(B|A)=P(AB)/P(A)=P(B)/P(A)=10/37.$\\
作业: $p_{1\dots5} = 0,\frac 1 4,\frac 1 2,\frac 3 4, 1$ 代表第 $i$ 枚硬币出现正面的概率. 
随机选取一枚硬币, 投掷直到出现正面. 求 $\Pr{C_i \mid B_4}$, $B_4$ 表示第 4 次首次出现正面. 
由贝叶斯, 
$\Pr{C_i | B_4} = $ $\frac {\Pr{B_4 | C_i} \Pr{C_i}} {\sum_{i \in [5]} \Pr{B_4 | C_i} \Pr{C_i}} = \frac {\Pr{B_4 | C_i}} {\sum_{i \in [5]} \Pr{B_4 | C_i}}.$
\section{随机变量}
随机变量 $X:\Omega \mapsto \mathbb R$ 对每个样本赋予实值. \\
CDF $F_X(x)=\Pr{X\leq x}$\\
PDF 1: $f_X(x) \geq 0,$
2: $\int_{-\infty}^{+\infty} f_X(x) \d x = 1,$
3: $\P{a < X < b} = \int_a^b f_X(x) \d x.$\\
$F^{-1}(q) = \inf\{x : F(x) > q\},(\max x: f(x) \leq q)$\\
Poisson 分布: 平均 5 分钟 10 人到店, 问等待第一个顾客两分钟以上的概率. 
两分钟实际到店人数 $X$ 服从参数为 $\lambda=4$ 的 Poisson 分布, 
$P(T>2)=P(X=0)=e^{-4}.$ 推广, 等待时间 $F(t) = 1 - e^{-\lambda t}, f(t) = F'(t) = \lambda e ^ {-\lambda t}$, 为指数分布, $t \in [0, \infty)$. \\
二维正态分布: 
$f(x, y) = \frac{1}{2\pi\sigma_1\sigma_2\sqrt{1-\rho^2}}\cdot \exp$ $ \left(-\frac{1}{2(1-\rho^2)}\left[\frac{(x-\mu_1)^2}{\sigma_1^2}+\frac{(y-\mu_2)^2}{\sigma_2^2}-\frac{2\rho(x-\mu_1)(y-\mu_2)}{\sigma_1\sigma_2}\right]\right)$
边际分布: $f_X(x)=\int_{-\infty}^{+\infty}f(x,y)\d y$\\
独立定义 $P(X\in A,Y \in B) = P(X \in A)P(Y \in B)$,
$f_{X,Y}(x,y)=f_X(x)f_Y(y)$, 独立iff $\forall x, y, f(x, y) = r(x)g(y), f_X(x)=r(x)/\int_{-\infty}^{\infty} r(x) \d x$\\
条件分布 $f_{X|Y}(x|y)= \frac{P(X = x, Y = y)}{P(Y = y)}=\frac{f_{X,Y}(x,y)}{f_Y(y)}$\\

\newcolumn
多项分布: $P(\dots,X_k=n_k)=\frac{n!}{n_1!\dots n_k!}p_1^{n_1}\dots p_k^{n_k}$\\
多元正态分布: $f(x_1, \dots,x_n)=\frac{1}{(2\pi)^{n/2}|\Sigma|^{1/2}}\exp$ $\left(-\frac{1}{2}(x-\mu)^T\Sigma^{-1}(x-\mu)\right)$\\
随机变量的函数: 对于随机变量 $X$ 考虑函数形式 $r(x)$, 计算 $Y=r(X)$ 的分布: 
对每个 $y$ 求集合 $A_y = \{x : r(x) \leq y\}$;
求 CDF: $F_{r(X)}(y) = \Pr{r(X) \leq y} = \Pr{X \in A_y} = \int_{A_y} f_X(x) \d x$;求 PDF: $f_Y(y) =  F_Y'(y)$. 当 $r$ 单调时, $r$ 的反函数为 $s = r^{-1}$, 有 $f_Y(y) = f_X(s(y)) \left|\frac{\d s(y)}{\d y}\right|$. \\
例: $f_X(x) = e ^ {-x} (x > 0)$, $F_X(x) = \int_{0}^{x} f_X(s)\d s = 1 - e ^ {-x}$. 令 $Y = r(X) = \log X, A_y = \{x : x \leq e ^ y\}$, $F_Y(y) = \Pr{X \leq e ^ y} = F_X(e ^ y) = 1 - e ^ {-e ^ y}$, $f_Y(y) = F_Y'(y) = e ^ y e ^ {-e ^ y}$. \\	
作业: 令 $X, Y \sim \mathrm{Uniform} (0, 1)$ 且独立, 求 $X - Y$ 的 PDF. 令 $Z = X - Y$, 
$ F_{Z} (z) = \int_{x + y \leq z} f(x, y) \d x \d y =
\begin{cases}
0 & \text{if $z \leq -1$} \\
\frac{(1 + a)^2}{2} & \text{if $-1 < z < 0$} \\
1 - \frac{(1 - a)^2}{2} & \text{if $0 \leq z < 1$} \\
1 & \text{if $z \geq 1$}
\end{cases}$ $\begin{cases}
f_Z(z) = \\
1 + z, \quad \text{$-1 < z < 0$} \\
1 - z, \quad \text{$0 \leq z < 1$} \\
0, \quad \text{otherwise}
\end{cases}
$
\section{期望}
$\mu_X = E(X) =  \int x \d F(x) < \infty$, 称期望存在.  \\
懒惰统计学家法则 $E(r(X)) = \int r(x) d F_X(x).$ \\
$k$ 阶矩 $E(X^k)$, 中心矩 $E((X - \mu)^k)$.  \\
$\sigma ^ 2 = V(X) = E((X - \mu)^2) = E(X^2) - E(X)^2$ \\
样本 $\overline X = 1/n \sum X_i, S_n^2 = 1/(n-1) \sum (X_i - \overline X)^2$ \\
定理: $E(\overline X_n) = \mu, V(\overline X_n) = \sigma ^ 2 / n, E(S_n^2) = \sigma^2.$
$Cov(X, Y) = E((X - \mu_X)(Y - \mu_Y))=E(XY)-E(X)E(Y)$, 不相关: $Cov(X_i, X_j) = 0$, 此时两者方差可加, 但不代表独立(例: $X=U(-1, 1), Y = |X|$)\\

\newcolumn

矩母函数 $M_X(t) = E(e^{tX})$. 如果 $Y = aX + b$, 则 $M_Y(t) = e^{bt} M_X(at)$. 
意义: $E(X ^ n) = M_X^{(n)} (0)$. \\
例: $X\sim \mathrm{Exp}(\lambda), E(X) = M_X'(0) = {\lambda \over (\lambda - t) ^ 2} |_{t = 0} = \frac 1 \lambda$, $E(X^2) = M_X^{(2)}(0)  = \frac {2 \lambda} {(\lambda - t) ^ 3} |_{t = 0} = \frac {2} {\lambda^2}.$
\section{不等式}
Markov $\Pr{X \geq a} \leq \frac {\E{X}} {a}, X \geq 0, a > 0$ \\
Chebyshev $\Pr{|X - \E{X}| \geq a} \leq \frac {\Var{X}} {a^2}$ \\
Mill $Z \sim N(0, 1)$, $\Pr{|Z| \geq t} \leq \frac 2 {\sqrt{2\pi}} \frac {e^{-t^2/2}} {t}$ \\
Hoeffding : \\
$\Pr{|X - \mu| \geq t} \leq 2 \exp \left( - \frac {2t^2} {\sum_{i = 1}^n(b_i - a_i)^2}\right)$ \\
$\E{e^{\alpha X}} \leq \exp \left(\frac {a^2(b - a)^2} {8}\right), \E{X} = 0, \alpha \in \mathbb R$ \\
例: $X \sim \mathrm{Bernoulli}(n, p)$,\\
$P(|\overline X_n - p| > \epsilon) \leq 2 \exp \left( - {2n\epsilon^2} \right)$ \\
% Chernoff $\Pr{X \geq (1 + \delta)\mu} \leq \left( \frac {e ^ {\delta}}  {(1 + \delta)^{1 + \delta}}\right)^\mu (\delta > -1)$ \\
% $\Pr{X \geq (1 + \delta)\mu} \leq \exp \left( - \frac {\delta ^ 2} {3} \mu \right)$\\
% $\Pr{X \geq (1 - \delta)\mu} \leq \exp \left( - \frac {\delta ^ 2} {2} \mu \right)$\\
Cauchy-Schwartz $E(XY) \leq \sqrt {E(X^2)E(Y^2)}$ \\
Jensen $E(g(X)) \geq g(E(X))$, $g$ 上凸. \\
\section{随机变量的收敛}
概率 $X_n \xrightarrow{P} X: P(|X_n - X| \geq \varepsilon) \rightarrow 0$, as $n \rightarrow \infty$.\\
分布 $X_n \rightsquigarrow X: F_{X_n}(x) \rightarrow F_X(x)$, $\forall x \in $ 连续点.\\
均方 $X_n \xrightarrow{qm} X : E(X_n - X)^2 \rightarrow 0$.\\	
均方 $\rightarrow$ 概率 $\xrightarrow{\leftarrow \textup{单点}}$ 分布.\\
依概率但不均方收敛 $X_n = \sqrt{n} I_{(0, 1/n)}(U(0, 1)); 0$\\
依分布但不依概率收敛 $X \sim N(0, 1), X_n = -X$\\
1. $X_n + Y_n \rightarrow X + Y: (P, qm)$
2. $X_n Y_n \rightarrow X Y : (P)$ 
3. $g(X_n)\rightarrow g(X) : (P, \rightsquigarrow)$
4. (Slutzky) $X_n \rightsquigarrow X, Y_n \rightsquigarrow c: X_n + Y_n \rightsquigarrow X + c, X_n  Y_n \rightsquigarrow cX$.\\
注: 通常 $X_n \rightsquigarrow X, Y_n \rightsquigarrow Y \nRightarrow X_n + Y_n \rightsquigarrow X + Y$. \\

\newcolumn
$\{X_i\}$ i.i.d., $\mu, \sigma^2$ 存在, $\overline X_N = \frac 1 n \sum_{i = 1}^n X_i$ \\
弱 LLN: $\overline X_n \xrightarrow{P} \mu : \lim_{n \rightarrow \infty} P(|\overline X_n - \mu| \leq \varepsilon) = 1$ \\
强 LLN: $\overline X_n \xrightarrow{a.s.} \mu : P(\lim_{n \rightarrow \infty} |\overline X_n - \mu| \leq \varepsilon) = 1$\\
CLT: $Z_n \equiv \frac{\sqrt{n} (\overline X_n - \mu)}{\sigma} \rightsquigarrow N(0, 1)$, i.e., \\
$\lim_{n \rightarrow \infty} P(Z_n \leq z) = \Phi(z) = \frac{1}{\sqrt{2\pi}} \int_{-\infty}^{z} e^{-t^2/2} \d t$\\
变形: $\overline X_n \approx N(\mu, \frac{\sigma^2}{n}), \sqrt{n} (\overline X_n - \mu ) \approx N(0, \sigma^2)$.\\
样本 $S_n^2 = (n - 1)^{-1}\sum(X_i - \overline X_n)^2$, CLT $\sigma$ 换为 $S_n$.\\
Delta 方法: 求极限分布 \\
$Y_n \sim N(\mu, \frac{\sigma ^ 2}{n}) \Rightarrow g(Y_n) \sim N(g(\mu), (g'(\mu))^2 \frac {\sigma ^ 2} {n} )$
\section{模型、统计推断与学习}
参数模型: $\mathcal{F}=\{f(x, \theta) : \theta \in \Theta\}$, 如正态 $\theta = (\mu, \sigma)$\\
非参数模型: 无法用有限参数表示, 回归/聚类/决策树等\\
统计量: 完全基于样本所得的量, 是样本的函数. \\
点估计: $\hat \theta = g(X_1, \dots, X_n)$. 
偏差 $\mathrm{bias}(\hat \theta_n) = E_\theta(\hat \theta_n) - \theta$. \\
无偏: $\mathrm{bias} = 0$; 相合: $\hat \theta_n \xrightarrow{P} \theta$. \\
标准误差 $\mathrm{se}(\hat \theta_n) = \sqrt{V(\hat \theta_n)}$; \\
MSE $ = E_\theta(\hat \theta_n - \theta)^2 = \mathrm{bias}^2 + \mathrm{se}^2$. \\
如果 $\mathrm{bias}\xrightarrow{P} 0$ 且 $\mathrm{se}\xrightarrow{P} 0$, 则 $\hat \theta_n \xrightarrow{P} \theta$. \\	
渐进正态性: $\frac{\hat\theta_n - \theta}{\mathrm{se}} \rightsquigarrow N(0, 1).$ \\
置信集: 置信区间 $C_n = (a_n, b_n)$, $P_\theta(\theta \in C_n) = 1 - \alpha$. \\
注: 置信区间是随机区间, 意思是给定若干组样本, 每组得到的置信区间覆盖真实参数的概率为 $1 - \alpha$. \\
基于正态的置信区间: $\hat \theta_n \pm z_{\alpha/2} \mathrm{se}(\hat \theta_n)$. \\
例: $\{X_n\} \sim \mathrm{Bernoulli}(p)$, 
由 Chebyshev $P(|\overline X_n - p| \geq \epsilon) \leq 2 \exp(-2n\varepsilon^2)$, 令 $\alpha= 2\exp(-2n\varepsilon) \Rightarrow \varepsilon ^ 2 = \log(2/a)/(2n)$, 因此 $C_n = (\overline X_n - \sqrt{\log(2/a)/(2n)}, \overline X_n + \sqrt{\log(2/a)/(2n)})$;
由渐进正态, $C_n =(\overline X_n - z_{\alpha/2} \sqrt{\overline X_n(1 - \overline X_n)/n}, \overline X_n + z_{\alpha/2} \sqrt{\overline X_n(1 - \overline X_n)/n})$. \\

\end{multicols}
\newpage
\begin{multicols}{4}

$z_{\alpha/2} = \Phi^{-1}(1 - \alpha / 2)$. $z_{0.05/2} = 1.96, z_{0.1/2} = 1.65, z_{0.025/2} = 2.24, z_{0.01/2}=2.58, z_{0.005/2}=2.80$. \\
假设检验: $H_0: \theta \in \Theta_0, H_1: \theta \in \Theta_1$. 一般把可以否定, 且可根据其构建分布机制的命题作为原假设. \\
显著性水平: 小概率事件发生的概率 $\alpha$; \\
临界值: $C$, 使得 $P_\theta(\text{拒绝 $H_0$, 如 $|\overline X - \mu| > C$}) = \alpha$.\\
拒绝域: $\mathcal{W}$, 如 $\{ (X_1, \dots, X_n) : |\overline X - \mu| > C\}$.\\
\section{CDF 和统计泛函的估计}
经验分布函数 $F_n(x) = \frac{1}{n} \sum_{i = 1}^n I(X_i \leq x)$
无偏,\\ $V(\hat F_n(x)) = \frac{F(x)(1 - F(x))}{n}$, $MSE \rightarrow 0$, $\xrightarrow{P}$,
$F \in (0,$ $1): \sqrt{n} (\hat F_n(x) - F(x)) \rightsquigarrow N(0, F(x)(1 - F(x)))$\\ 
统计泛函: $T(F)$ 是分布函数 $F$ 的函数. \\
嵌入式估计量: $\hat \theta_n = T(\hat F_n)$. \\
线性泛函: $T(F) = \int_{-\infty}^{+\infty} r(x) \d F(x)$, \\
满足 $T(aF + bG) = aT(F) + bT(G)$, 嵌入式估计量: $T(\hat F_n) = \int_{-\infty}^{+\infty} r(x) \d \hat F_n(x)$ $= \frac{1}{n} \sum_{i = 1}^n r(X_i)$,\\
近似 $1-\alpha$ 置信区间为 $T(\hat F_n) \pm z_{\alpha/2} \hat {\mathrm{se}}$.\\
例: $\hat \mu = \overline X_n, \hat \sigma^2 = \frac{1}{n} \sum_{i = 1}^n X_i^2 - \overline X_n^2$. \\
\section{Bootstrap 方法}
从 $\hat F_n$ 中生成 $X_1^*, \dots, X_n^*$ 计算统计量, 重复 $B$ 次. \\
方差估计:  $V_{\mathrm{boot}} = \frac{1}{B} \sum_{b = 1}^B (T_{n,b}^* - \overline T^*)^2$\\
正态区间法 $T_n \pm z_{\alpha/2} \sqrt{V_{\mathrm{boot}}}$ , 除非接近正态否则不准确. \\
枢轴量法: 定义 $R_n = \hat \theta - \theta$, $\hat \theta_{n,1 \dots B }^*$ 为副本, \\
$\theta_\beta^*$ 为 $\beta$ 分位数, $C_n = (2 \hat \theta_n - \theta_{1 -\alpha / 2}^*, 2 \hat \theta_n - \theta_{\alpha / 2}^*)$\\
分位数置信区间: $(\theta_{\alpha/2}^*, \theta_{1 - \alpha/2}^*)$ \\
\section{参数推断}
似然函数: $\mathcal{L}_n(\theta) = \prod_{i = 1}^n f(X_i, \theta)$,\\
对数似然函数 $\ell_n(\theta) = \log L_n(\theta)$\\
极大似然估计 $\hat \theta_n$: 使 $L_n(\theta)$ 达到最大值的 $\theta$ 的值,\\
可以解方程 $\partial \ell_n(\theta) / \partial \theta = 0$ 得到. \\
例 1.正态分布 $\hat \mu = \overline X, \hat \sigma = S.$;
2. $\{X_i\} \sim U(0, \theta)$ i.i.d.,
$f(x; \theta) = 1/\theta$ if $0\leq x\leq \theta$,
$\mathcal{L}_n(\theta) = (1/\theta)^n$ if $\theta \geq \max \{X_i\}$,
因此 $\hat \theta_n = \max \{X_i\}$
3. $f(x) = \theta x ^ {\theta - 1}$ 求 $\hat \theta$, $\ell_n(\theta) = n \log \theta + (\theta - 1) \sum \log X_i$, 
解方程 $\partial \ell_n(\theta) / \partial \theta = 0$ 得到
$\hat \theta_n = - \frac{n}{\sum \log X_i}$.\\
极大似然估计的相合性: $\hat \theta_n \xrightarrow{P} \theta_*$
KL 距离: $D(P, Q) = \int p(x) \log \frac{p(x)}{q(x)} \d x$, $D(f, g)\geq 0, D(f, g) = 0 \Leftrightarrow f = g$.\\
模型可识别: $\theta \neq \phi \Rightarrow D(\theta, \phi) > 0$\\
极大似然估计的同变性: $\tau = g(\theta), \hat \tau_n = g(\hat \theta_n)$\\
Fisher $I_n(\theta) = V_\theta\left( \sum s(X_i; \theta) \right) = \sum V_\theta(s(X_i; \theta)) = E(s^2(X; \theta)) = n E(s(x_i; \theta))$, $s(X; \theta) = \frac {\partial \log f(X; \theta)} {\partial \theta}$\\
渐进正态性: 令 $\mathrm{se} = \sqrt{V(\hat \theta_n)}$, 适当正则条件下 \\

\newcolumn
1. $\mathrm{se} \approx \sqrt{1 / I_n(\theta)}, \frac{\hat \theta_n - \theta}{\mathrm{se}} \rightsquigarrow N(0, 1)$, \\
2. $\hat {\mathrm{se}} = \sqrt{1 / I_n(\hat \theta_n)}, \frac{\hat \theta_n - \theta}{\hat {\mathrm{se}}} \rightsquigarrow N(0, 1)$\\
3. $C_n = \hat \theta_n \pm z_{\alpha/2} \hat {\mathrm{se}}, P_\theta(\theta \in C_n) \rightarrow 1 - \alpha$\\
例: $N(\theta, \sigma ^ 2)$,
$f(X; \theta) = \frac{1}{\sqrt{2\pi\sigma^2}} \exp \left( - \frac{(X - \theta)^2}{2\sigma^2} \right)$,\\
记分函数 $s(X; \theta) = \frac{\partial \log f(X; \theta)}{\partial \theta} = \frac{X - \theta}{\sigma^2}$,
$I_n(\theta) = $ $n E(s^2(X; \theta)) = -n E_\theta(s'(X; \theta)) = -n E_\theta(-\frac 1 {\sigma^2}) = \frac {n} {\sigma^2}$,
$\hat {\mathrm{se}} = \sqrt{\frac{\sigma^2}{n}}$, 故 $\overline X$ 近似服从 $N(\theta, \sigma^2 / n)$. \\
极大似然估计 Delta 方法: 如果 $\tau = g(\theta), g'(\theta) \neq 0$, 则 $\frac {\hat \tau_n - \tau} {\hat {\mathrm{se}}(\hat \tau_n)} \rightsquigarrow N(0, 1)$,
其中 $\hat \tau_n = g(\hat \theta_n)$, 且 $\hat {\mathrm{se}}(\hat \tau_n) = \sqrt{V(\hat \tau_n)} = |g'(\hat \theta_n)| \hat {\mathrm{se}}(\hat \theta_n)$. \\

例: $\mathrm{Poisson}(\lambda)$, 矩估计: $\hat \lambda = \overline X_n$,
$f(X; \lambda) = \frac{\lambda^x}{x!} e^{-\lambda}$,
$\mathcal{L}_n = \prod \frac{\lambda^{X_i}e^{-\lambda}}{(X_i)!}$, $\ell_n = \sum (X_i \log \lambda - \lambda - \log (X_i)!$), $\partial = 0$ 得 $\overline X$.
$s(X; \lambda) = \frac{\partial \log f(X; \lambda)}{\partial \lambda} = \frac{X - \lambda}{\lambda}$,
$I_n(\lambda) = n I(\lambda) = n E(s^2(X; \lambda)) = \frac{n}{\lambda}$, $\hat {\mathrm{se}} = \sqrt{\frac{\lambda}{n}}$, 
$\overline X$ 近似服从 $N(\lambda, \lambda / n)$,
$1 - \alpha$ 置信区间为 $\hat \lambda \pm z_{\alpha/2} \sqrt{\frac {\hat \lambda} {n}}$. \\
例: $N(\theta, 1)$, $Y_i = I(X_i > 0)$,$\psi = P(Y_1 = 1)$.
$\mathcal{L}_n(\theta) = \prod \frac{1}{\sqrt{2\pi}} e^{-\frac{(X_i - \theta)^2}{2}}$,
$\ell_n(\theta) = -\frac{n}{2} \log 2\pi - \frac{1}{2} \sum (X_i - \theta)^2$,
$\partial \ell_n(\theta) / \partial \theta = \sum (x_i - \theta), = 0 : \hat \theta = \overline X$.
$I_n(\theta) = n E_\theta(s^2(X_i; \theta)) = n$,
$\hat{\mathrm{se}}(\hat \theta) = \frac{1}{\sqrt{I_n(\hat \theta)}} = \frac{1}{\sqrt{n}}$,
$\psi = \Phi(\theta)$, $\hat \psi = \Phi(\hat \theta) = \Phi(\overline X)$,
$\hat{\mathrm{se}}(\hat \psi) = |\Phi'(\hat \theta)| \hat{\mathrm{se}}(\hat \theta) = \phi(\hat \theta) \hat{\mathrm{se}}(\hat \theta)$.\\
$\tilde \psi = 1/n \sum Y = E[\overline Y] \xrightarrow{P} E[Y] = \psi$, 渐进效率 $V(\tilde \psi) / V(\hat \psi) = \psi (1 - \psi) / \phi(\theta) = \Phi(\theta) (1 - \Phi(\theta)) / \phi(\theta)$. 
非正态: 由 LLN $\hat \psi \xrightarrow{P} \Phi(\mu)$, $F_X(0) \neq 1 - \Phi(\mu)$ 都不相合. \\
例: $X_1 \sim \mathrm{Binomial}(n_1, p_1), X_2 \sim \mathrm{Binomial}(n_2, p_2), \psi = p_1 - p_2$,
$f(X_i; p_i) = \binom{n_i}{X_i} p_i^{X_i} (1 - p_i)^{n_i - X_i}$,
$\frac {\partial} {\partial p_i} s(X_i; p_i) = \frac {X_i} {p_i} + \frac {X_i} {1 - p_i} = \frac {X_i - n p_i } {p_i (1 - p_i)}$
$\hat p_i = X_i / n_i, \hat \psi = \hat p_1 - \hat p_2 = X_1/n_1 - X_2 / n_2$.
\begin{tiny}
$I(p_1, p_2) = (\partial ^ 2 \log f((X_1, X_2); \psi) / \partial p_i \partial p_j) = \begin{bmatrix}
	\frac{n_1}{p_1(1 - p_1)} & 0\\
	0 & \frac{n_2}{p_2(1 - p_2)}
	\end{bmatrix}. $
\end{tiny} \\
多参: $\mathbf{\theta} = (\theta_1, \dots, \theta_k)^T$ 
Fisher 信息矩阵 $I_n(\mathbf{\theta}) = $ $(E_{\mathbf{\theta}}(H_{ij}))$,
$ H_{ij} = \frac{\partial \log f(X; \mathbf{\theta})}{\partial \theta_i \partial \theta_j}$,
$J_n(\mathbf{\theta}) = I_n^{-1}(\mathbf{\theta})$\\
多参数 Delta 方法: $\nabla g = (\frac{\partial g}{\partial \theta_1}, \dots)^T$ 在 $\hat \theta_n$ 处不为 $0$, 令 $\hat \tau_n = g(\hat \theta_n)$, 则
$\frac {\hat \tau_n - \tau} {\sqrt{\hat \tau_n^T J_n(\hat \theta_n) \hat \tau_n}} \rightsquigarrow N(0, 1)$,
其中 $\hat {\mathrm{se}}(\hat \tau_n) = \sqrt{(\nabla g |_{\theta = \hat \theta_n}^T) J_n(\hat \theta_n) (\nabla g |_{\theta = \hat \theta_n})}$. \\
例: $X_1\dots X_n \sim N(\mu, \sigma^2)$, $\psi = g(\mu, \sigma) = \sigma / \mu$,
$(\mu, \sigma ^ 2)$ MLE: $(\overline X, S^2)$,\begin{scriptsize}
$I_n(\mu, \sigma^2) = \begin{pmatrix}
	n / \sigma^2 & 0 \\
	0 & 2n / \sigma^2
\end{pmatrix}$,$J_n = \begin{pmatrix}
	\sigma^2 / n & 0 \\
	0 & \sigma^2 / 2n
\end{pmatrix}$, $\nabla g = (-\frac{\sigma}{\mu^2}, \frac{1}{\mu})^T$,
$\hat {\mathrm{se}}(\hat \tau) = \frac 1 {\sqrt n} \sqrt{\frac {\hat \sigma ^ 4} {\hat \mu^4} + \frac {\hat \sigma^2} {2 \hat \mu^2}}$.
\end{scriptsize} \\

\newcolumn

\section{假设检验与 $p$ 值}
检验统计量: $T_n = T(X_1, \dots, X_n)$, \\
拒绝域 $\mathcal{A} = \{T_n > c\}$, $c$ 为临界值. \\
第一类错误: 假阳性;第二类错误: 假阴性. \\
势函数 $\beta(\theta) = P_\theta(X\in \mathcal{A})$, 容度 $\beta = \sup_{\theta \in \Theta_1} \beta(\theta)$\\
容度 $\leq \alpha$ 称检验水平为 $\alpha$. 简单 $\theta = \theta_0$ 复合 $\theta \geq \theta_0$ \\
双边 $H_0 : \theta = \theta_0$ vs \dots
单边 $H_0 : \theta \geq \theta_0$ vs \dots\\
例: $X_1 \dots X_n \sim N(\mu, \sigma^2)$, 检验 $H_0 : \mu = \mu_0$ vs $H_1 : \mu \neq \mu_0$,
检验统计量 $T_n = \sqrt{n} \frac{|\overline X_n - \mu_0|}{S_n}$, \\
拒绝域 $\{T_n > z_{\alpha/2}\}$,
势函数 $\beta(\mu) = P_\mu(T_n > z_{\alpha/2})$\\
如检验 $H_0: \mu \leq 0$ vs $H_1: \mu > 0$,
拒绝域 $\mathcal{A} = \{\overline X > c\}$,
势函数 $\beta(\mu) = P_\mu(\overline X > c) = P(\sqrt{n}\frac {(\overline X - \mu)} {\sigma} > \sqrt{n} \frac {c - \mu} {\sigma}) = 1 - \Phi(\sqrt{n} \frac {c - \mu} {\sigma})$,
检验水平 $\alpha = \sup_{\mu \leq \mu_0} \beta(\mu) = 1 - \Phi(\sqrt{n} \frac {c - \mu_0} {\sigma})$,
设定显著性水平 $\alpha$, $c=\mu_0 + \frac{\sigma}{\sqrt{n}} z_{\alpha / 2}$. \\
Wald 检验: $H_0: \theta = \theta_0$ vs $H_1: \theta \neq \theta_0$,
若 $\hat \theta$ 渐进正态, $W = \frac{\hat \theta - \theta_0}{\hat {\mathrm{se}}(\hat \theta)}$, 当 $|W| > z_{\alpha / 2}$ 时拒绝 $H_0$. 势函数: 

$\beta(\theta_*)\approx
1 - \Phi(z_{\alpha / 2} + \frac{\theta_0 - \theta_*}{\hat {\mathrm{se}}(\hat \theta)})
+\Phi(-z_{\alpha / 2} + \frac{\theta_0 - \theta_*}{\hat {\mathrm{se}}(\hat \theta)})$\\
例: 给定样本 $\{X_m\}, \{Y_n\}$, 检验均值是否相等. \\
$H_0: \mu_X - \mu_Y = 0$ vs $H_1: \mu_X - \mu_Y \neq 0$, \\
$V(\hat \theta) = V(\overline X - \overline Y) = \frac{\sigma_X^2}{m} + \frac{\sigma_Y^2}{n}$, 
$\hat {\mathrm{se}} = \sqrt{\frac{S_X^2}{m} + \frac{S_Y^2}{n}}$, \\
拒绝域为
$|\overline X - \overline Y| > z_{\alpha / 2} \hat {\mathrm{se}}$, \\
例:给定样本 $\{(X_n, Y_n)\}$ 检验均值是否相等. \\
$H_0: \mu_X = \mu_Y$ vs $H_1: \mu_X \neq \mu_Y$, 构造 $\theta = \mu_1 - \mu_2$, \\
估计为 $\hat \theta = \overline X - \overline Y$.
方差: $V(\hat \theta) =  \sigma^2/n$, 其中 $\sigma$ 是 $X_i - Y_i$ 总体标准差, 用样本标准差估计
$\hat{\mathrm{se}} = \sqrt{S^2/n}$, 拒绝域为 $|\overline X - \overline Y| > z_{\alpha / 2} \hat {\mathrm{se}}$. \\
Wald 置信区间: $C = \left(\hat \theta - z_{\alpha / 2} \hat {\mathrm{se}}(\hat \theta), \hat \theta + z_{\alpha / 2} \hat {\mathrm{se}}(\hat \theta)\right)$, \\
p 值: $\inf \{\alpha : \theta_0 \in C\}$, 可以拒绝 $H_0$ 的最小检验水平. \\
例: 计算出了 $W = \left| \frac{\overline X - \overline Y}{\hat {\mathrm{se}}}\right|$, p 值: $P(|Z| > W)$\begin{tiny}($Z\sim N(0, 1)$)\end{tiny} \\

\newcolumn

Pearson $\chi^2$ 统计量: $X= (X_1, \dots, X_k)$ 服从多项分布 $M(n, p=(p_1, \dots, p_k))$,
检验 $H_{0/1}: p ?= p_0$. \\
检验统计量 $\chi^2 = \sum_{i = 1}^k \frac{(X_i - np_{0i})^2}{np_{0i}} = \sum_{i=1}^k \frac{(X_j - E_j)^2}{E_j}$.
在 $H_0$ 下 $T \rightsquigarrow \chi^2_{k - 1}$, 拒绝域为 $\{T > \chi^2_{\alpha}(k - 1)\}$,\\
p 值为 $P(\chi^2(k - 1) > T)$. \\
置换检验: 检验分布是否相同, shuffle 看统计量分位数\\
似然比检验: $H_0: \theta \in \Theta_0$ vs $H_1: \theta \notin \Theta_0$, $\lambda = 2\log\left( \frac {L(\hat \theta)} {L(\hat {\theta_0})}\right)$,
$\hat \theta_0$ 表示限制 $\Theta_0$.
$\lambda \rightsquigarrow \chi^2_{\dim \Theta - \dim \Theta_0}$, 拒绝域 $\{\lambda > \chi^2_{\alpha}(\dim \Theta - \dim \Theta_0)\}$,
p 值 $P(\chi^2(\dots) > \lambda)$. \\
例: 豌豆实验自由度为 $3-0$ 而非 $4$, 因为概率和为 $1$. \\
多重假设检验: Bonferroni: $\alpha / m$ (Union bound).
BH: 排序后 $p_i \leq \frac{i \alpha}{m C_m} $, $C_m$ 不独立 $\sum_{i = 1}^m \frac{1}{i}$, 独立 $1$. \\
\setcounter{section}{12}
\section{线性回归和 Logistic 回归}
简单线性回归: $Y = \beta_0 + \beta_1 X + \epsilon$, $\epsilon : E=0, V=\sigma^2$,
$\hat y_i = \hat \beta_0 + \hat \beta_1 x, \hat \epsilon = y_i - \hat y_i$.\\
最小二乘: 使 $\mathrm{RSS} = \sum \hat \epsilon_i^2$ 最小, $E(\mathrm{RSS}) = (n - 2)\sigma^2$ \\
$\ell_{xy}= \sum (x_i - \overline x)(y_i - \overline y) = \sum x_i y_i - n \overline x \overline y$, \\
$\ell_{xx} = \sum (x_i - \overline x)^2 = \sum x_i^2 - n \overline x^2$, \\
$\hat \beta_1 = \ell_{xy} / \ell_{xx}$, $\hat \beta_0 = \overline y - \hat \beta_1 \overline x$, $\hat \sigma ^ 2$ 无偏估计为 $\frac{1}{n - 2} \mathrm{RSS}$. \\
若 $\epsilon_i | x_i \sim N(0, \sigma^2)$, 则 \\
$L(\beta_0, \beta_1, \sigma^2) \propto \sigma^{-n} \exp \left( - \frac{1}{2\sigma^2} \sum (y_i - \beta_0 - \beta_1 x_i)^2 \right)$,
$\ell(\beta_0, \beta_1, \sigma^2) \propto -n \log \sigma - \frac{1}{2\sigma^2} \sum (y_i - \beta_0 - \beta_1 x_i)^2$.
正态性假设下, 最小二乘即极大似然. $\hat \sigma^2 = 1/n \sum \hat \epsilon_i^2$.
$\hat {\mathrm{se}}(\hat \beta_1) = \sqrt{\frac{\hat \sigma^2}{\ell_{xx}}}$,
$\hat {\mathrm{se}}(\hat \beta_0) = \hat {\mathrm{se}}(\hat \beta_1) \sqrt{\frac{1}{n} \sum x_i^2}$,
适当条件下, 相合, 渐进正态 ($\frac{\hat \beta_i - \beta_i}{\hat{\mathrm{se}}(\hat\beta_i)}) \rightsquigarrow N(0, 1)$,
渐进置信区间 $\hat \beta_i \pm z_{\alpha/2} \hat{\mathrm{se}}(\hat\beta_i)$, Wald 检验 $W=\hat \beta_1 / \hat{\mathrm{se}}(\hat\beta_1)$. \\
近似预测区间: $\hat \xi_n^2 = \hat \sigma^2 \left( \frac{1}{n} + \frac{(x_0 - \overline x)^2}{\ell_{xx}} \right)$, $\hat y_* \pm z_{\alpha/2} \hat \xi_n$. \\
Logistic 回归: 分类问题损失函数不连续, 接一个 Sigmoid 函数 (Logistic) $g(x) = \frac{1}{1 + e^{-x}}$. 没有显式解. \\
\end{multicols}
	
\end{document}