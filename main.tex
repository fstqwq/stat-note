\documentclass[titlepage, a4paper, landscape]{article}


\usepackage{xeCJK}

\setCJKmainfont{Source Han Serif SC}[Scale=0.8]
\setCJKmonofont{SimHei}[Scale=0.8]
\setCJKsansfont{KaiTi}[Scale=0.8]

\setmainfont{Linux Libertine O}[Scale=0.925]
\setmonofont{Consolas}[Scale=0.775]
%\setsansfont{Gill Sans Medium}

\XeTeXlinebreaklocale "zh"
\XeTeXlinebreakskip = 0pt plus 1pt

\setlength{\parindent}{0em}\setlength{\parskip}{1pt}
\setlength\itemsep{1pt}

\makeatletter
\renewcommand{\paragraph}{%
  \@startsection{paragraph}{4}%
  {\z@}{1pt \@plus 1pt \@minus 1pt}{-1em}%
  {\normalfont\normalsize\bfseries}%
}
\makeatother
\usepackage[lining,semibold,type1]{libertine} % a bit lighter than Times--no osf in math
\usepackage[T1]{fontenc} % best for Western European languages

\usepackage{amsthm,amsmath,amssymb}
\usepackage[table]{xcolor}
\usepackage[bookmarks=true,hypertexnames=false,pagebackref]{hyperref}
\hypersetup{colorlinks=true, citecolor=blue, linkcolor=red,
  urlcolor=blue}
\usepackage{textcomp} % required to get special symbols
\usepackage[varqu,varl]{inconsolata}% a typewriter font must be defined
\usepackage[libertine,vvarbb]{newtxmath}
\usepackage[scr=rsfso]{mathalfa}
\usepackage{bm}
\usepackage{cleveref}
\usepackage{graphicx}
\usepackage{booktabs}

\newtheorem{theorem}{Theorem}
\newtheorem{lemma}[theorem]{Lemma}
\newtheorem{corollary}[theorem]{Corollary}
\newtheorem{fact}[theorem]{Fact}
\newtheorem{definition}[theorem]{Definition}
\newtheorem{definitions}[theorem]{Definitions}
\newtheorem{conjecture}[theorem]{Conjecture}
\newtheorem{claim}[theorem]{Claim}
\newtheorem*{myclaim}{Claim}
\newtheorem*{remark}{Remark}
\newtheorem{proposition}[theorem]{Proposition}
\newtheorem{condition}{Condition}

\newcommand{\norm}[1]{\left\Vert#1\right\Vert}
\newcommand{\abs}[1]{\left\vert#1\right\vert}
\newcommand{\set}[1]{\left\{#1\right\}}
\newcommand{\tuple}[1]{\left(#1\right)} \newcommand{\eps}{\varepsilon}
\newcommand{\inner}[2]{\langle #1,#2\rangle} \newcommand{\tp}{\tuple}
\renewcommand{\mid}{\;\middle\vert\;} \newcommand{\cmid}{\,:\,}
\newcommand{\numP}{\#\mathbf{P}} \renewcommand{\P}{\mathbf{P}}
\newcommand{\defeq}{\triangleq} \renewcommand{\d}{\,\-d}
\newcommand{\ol}{\overline}

\usepackage{algorithm}
\usepackage{algorithmicx}
%\renewcommand{\thealgorithm}{} %% disable the algorithm counter
\usepackage{algpseudocode}


\def\*#1{\mathbf{#1}} \def\+#1{\mathcal{#1}} \def\-#1{\mathrm{#1}} \def\^#1{\mathbb{#1}} \def\$#1{\mathtt{#1}}


\usepackage{xifthen}
\renewcommand{\Pr}[2][]{ \ifthenelse{\isempty{#1}}
  {{P}\left(#2\right)}
  {\mathop{P}_{#1}\left(#2\right)} }
\renewcommand{\P}[2][]{ \ifthenelse{\isempty{#1}}
  {{P}\left(#2\right)}
  {\mathop{P}_{#1}\left(#2\right)} }
\newcommand{\E}[2][]{ \ifthenelse{\isempty{#1}}
  {{E}\left(#2\right)}
  {\mathop{E}_{#1}\left(#2\right)} }
\newcommand{\Var}[2][]{ \ifthenelse{\isempty{#1}}
  {{Var}\left(#2\right)}
  {\mathop{{Var}}_{#1}\left(#2\right)} }
  
\newcommand{\Cov}[2][]{ \ifthenelse{\isempty{#1}}
  {{Cov}\left(#2\right)}
  {\mathop{{Cov}}_{#1}\left(#2\right)} }

\usepackage[inner=0.6cm, outer=0.6cm, top=0.6cm, bottom=0.6cm]{geometry}
\usepackage{multicol}
\usepackage{titlesec}
%configure section style
\titleformat{\section}
{\Large\bfseries}			% The style of the section title
{\thesection.}				% a prefix
{0pt}						% How much space exists between the prefix and the title
{}					% How the section is represented
% \titleformat{\section}{\huge}{}{0pt}{}
\titlespacing{\section}{0pt}{0pt}{0pt}
\titlespacing{\subsection}{0pt}{0pt}{0pt}
\titlespacing{\subsubsection}{0pt}{0pt}{0pt}

\begin{document}
		\begin{tabular}{lllll}
			\multicolumn{5}{l}{{\Huge{基础数理统计 2023 Spring}}} \\
			\multicolumn{5}{l}{{\Huge{~}}} \\
			\toprule
			Discrete dist. & pmf & mean & variance & mgf/moment \\
			\midrule
			{Discrete Uniform}$(n)$ & $\frac{1}{n}$ & $\frac{n+1}{2}$ & $\frac{n^2-1}{12}$ & $\frac {e^t (1 - e^{n t}) } {n (1 - e^t) }=\frac 1n \sum e^{it}$ \\
			\rowcolor{gray!15}{Bernoulli}$(p)$ & $p^x(1-p)^{1-x}$ & $p$ & $p(1-p)$ & $(1-p) + pe^t$ \\
			{Binomial}$(n, p)$ & $\binom{n}{x}p^x(1-p)^{n-x}$ & $np$ & $np(1-p)$ & $((1-p) + pe^t)^n$ \\
			\rowcolor{gray!15}{Geometric}$(p)$ & $(1-p)^{x-1}p$ & $\frac{1}{p}$ & $\frac{1-p}{p^2}$ & $\frac{pe^t}{1-(1-p)e^t}$ \\
			{Poisson}$(\lambda)$ & $\displaystyle\frac{\lambda^x}{x!}e^{-\lambda}$ & $\lambda$ & $\lambda$ & $\displaystyle e^{\lambda(e^t-1)}$ \\
			%& \scriptsize{If $Y$ is Gamma($\alpha, \beta$), X is Poisson($\frac x  \beta$), and $\alpha$ is integer, then $P(X\geq \alpha) = P(Y\leq y)$.} &&& \\
			\rowcolor{gray!15}\shortstack{{Beta-binomial}\\$(n, \alpha, \beta)$}
			& \shortstack{$\binom{n}{x} \frac {\Gamma (\alpha + \beta)} {\Gamma (\alpha) \Gamma (\beta)} \cdot $ \\
			$\frac {\Gamma (x + \alpha) \Gamma (n - x + \beta)} {\Gamma (n + \alpha + \beta)}$}
			& $\displaystyle\frac{n \alpha}{\alpha+\beta}$
			& $\displaystyle\frac{n \alpha \beta} {(\alpha + \beta)^2}$ & 
			\scriptsize{\shortstack{If $X|P$ is Binomial$(n, P)$, \\and $P$ is Beta($\alpha, \beta$), then $X$ is \\Beta-binomial$(n, \alpha, \beta)$.}} \\
			\bottomrule
		\end{tabular}
		\begin{tabular}{lllll}
			\toprule
			Continuous dist. & pdf & mean & variance & mgf/moment \\
			\midrule
			{Uniform}$(a, b)$ & $\displaystyle\frac{1}{b-a}$ & $\displaystyle\frac{a+b}{2}$ & $\displaystyle\frac{(b-a)^2}{12}$ & $\displaystyle\frac{e^{tb}-e^{ta}}{t(b-a)}$ \\
			\rowcolor{gray!15}{Exponential}$(\theta)$ & $\displaystyle\frac{1}{\theta}e^{-\frac{x}{\theta}}$ & $\theta$ & $\theta^2$ & $\displaystyle\frac{1}{(1-t\theta)^2}, t < \frac{1}{\theta}$ \\
			\rowcolor{gray!15}{Exponential}$(\lambda)$ & $\lambda e^{-\lambda x}$ & $\displaystyle\frac{1}{\lambda}$ & $\displaystyle\frac{1}{\lambda^2}$ & $\displaystyle\frac{\lambda}{\lambda-t}, t < \lambda$ \\
			{Normal}$(\mu, \sigma^2)$ & \large{$\displaystyle\frac{1}{\sqrt{2\pi\sigma^2}}e^{-\frac{(x-\mu)^2}{2\sigma^2}}$} & $\mu$ & $\sigma^2$ & \Large{$\displaystyle e^{\mu t + \frac{\sigma^2 t^2}{2}}$} \\
			\rowcolor{gray!15}{Gamma}$(\alpha, \beta)$ & $\displaystyle\frac{1}{\Gamma(\alpha)\beta^\alpha}x^{\alpha-1}e^{-x / \beta}$ & $\alpha\beta$ & $\alpha\beta^2$ & $\displaystyle\left(\frac{1}{1-\beta t}\right)^\alpha, t < \frac{1}{\beta}$ \\
			{Beta}$(\alpha, \beta)$ & $\displaystyle \frac{\Gamma(\alpha+\beta)}{\Gamma(\alpha)\Gamma(\beta)}x^{\alpha-1}(1-x)^{\beta-1}$ & $\displaystyle \frac{\alpha}{\alpha+\beta}$ & $ \frac{\alpha\beta}{(\alpha+\beta)^2(\alpha+\beta+1)}$ & $1 + \sum_{k=1}^{\infty} \left(\prod_{i=0}^{k-1} \frac{\alpha+i}{\alpha+\beta+i}\right) \frac{t^k}{k!}$ \\
			\rowcolor{gray!15}{Cauchy}$(\mu, \sigma)$ & $\frac{1}{\pi \sigma}\frac{1}{1+\left(\frac{x-\mu}{\sigma}\right)^2}$ & undefined & undefined & undefined \\
			{$\chi^2_p$} & $\displaystyle \frac{1}{2^{p/2}\Gamma(p/2)}x^{p/2-1}e^{-x/2}$ & $p$ & $2p$ & $(1-2t)^{-p/2}, t < 1/2$ \\
			%\rowcolor{gray!15}{Student's t}$(\nu)$ & $\frac{\Gamma((\nu+1)/2)}{\sqrt{\nu\pi}\Gamma(\nu/2)}\left(1+\frac{x^2}{\nu}\right)^{-(\nu+1)/2}$ & $0, \nu > 1$ & $ \frac \nu {\nu - 2}, \nu > 2$ & $EX^{n=2k}: \frac {\Gamma((\nu + 1) / 2) \Gamma{\nu - n / 2}} {\sqrt{\pi} \Gamma(\frac \nu 2)} {\nu^{n/2}}$\\
			\bottomrule
		\end{tabular}
	\begin{multicols}{4}
		\section{概率}
条件概率 $P(A|B)=P(AB)/P(B)$\\
全概率 $P(B)=\sum_{i}P(B|A_i)P(A_i)$\\
贝叶斯 $P(A_i |B) = \frac{P(B|A_i)P(A_i)}{\sum_j P(B|A_j)P(A_j)}$\\
作业:蓝眼 $1/4$。对于三个孩子的家庭,如果至少有一个孩子是蓝眼睛(A),至少有两个蓝眼睛(B)的概率为
$P(B|A)=P(AB)/P(A)=P(B)/P(A)=10/37.$\\
作业:$p_{1\dots5} = 0,\frac 1 4,\frac 1 2,\frac 3 4, 1$ 代表第 $i$ 枚硬币出现正面的概率。
随机选取一枚硬币,投掷直到出现正面。求 $\Pr{C_i \mid B_4}$, $B_4$ 表示第 4 次首次出现正面。
由贝叶斯,
$\Pr{C_i | B_4} = $ $\frac {\Pr{B_4 | C_i} \Pr{C_i}} {\sum_{i \in [5]} \Pr{B_4 | C_i} \Pr{C_i}} = \frac {\Pr{B_4 | C_i}} {\sum_{i \in [5]} \Pr{B_4 | C_i}}.$
\section{随机变量}
随机变量 $X:\Omega \mapsto \mathbb R$ 对每个样本赋予实值。\\
CDF $F_X(x)=\Pr{X\leq x}$\\
PDF 1: $f_X(x) \geq 0,$
2: $\int_{-\infty}^{+\infty} f_X(x) \d x = 1,$
3: $\P{a < X < b} = \int_a^b f_X(x) \d x.$\\
$F^{-1}(q) = \inf\{x : F(x) > q\},(\max x: f(x) \leq q)$\\
Poisson 分布:平均 5 分钟 10 人到店,问等待第一个顾客两分钟以上的概率。
两分钟实际到店人数 $X$ 服从参数为 $\lambda=4$ 的 Poisson 分布,
$P(T>2)=P(X=0)=e^{-4}.$ 推广,等待时间 $F(t) = 1 - e^{-\lambda t}, f(t) = F'(t) = \lambda e ^ {-\lambda t}$,为指数分布,$t \in [0, \infty)$。\\
二维正态分布:
$f(x, y) = \frac{1}{2\pi\sigma_1\sigma_2\sqrt{1-\rho^2}}\cdot \exp$ $ \left(-\frac{1}{2(1-\rho^2)}\left[\frac{(x-\mu_1)^2}{\sigma_1^2}+\frac{(y-\mu_2)^2}{\sigma_2^2}-\frac{2\rho(x-\mu_1)(y-\mu_2)}{\sigma_1\sigma_2}\right]\right)$

\newcolumn
边际分布:$f_X(x)=\int_{-\infty}^{+\infty}f(x,y)\d y$\\
独立定义 $P(X\in A,Y \in B) = P(X \in A)P(Y \in B)$,
$f_{X,Y}(x,y)=f_X(x)f_Y(y)$, 独立iff $\forall x, y, f(x, y) = r(x)g(y), f_X(x)=r(x)/\int_{-\infty}^{\infty} r(x) \d x$\\
条件分布 $f_{X|Y}(x|y)= \frac{P(X = x, Y = y)}{P(Y = y)}=\frac{f_{X,Y}(x,y)}{f_Y(y)}$\\
多项分布:$P(\dots,X_k=n_k)=\frac{n!}{n_1!\dots n_k!}p_1^{n_1}\dots p_k^{n_k}$\\
多元正态分布:$f(x_1, \dots,x_n)=\frac{1}{(2\pi)^{n/2}|\Sigma|^{1/2}}\exp$ $\left(-\frac{1}{2}(x-\mu)^T\Sigma^{-1}(x-\mu)\right)$\\
随机变量的函数:对于随机变量 $X$ 考虑函数形式 $r(x)$,计算 $Y=r(X)$ 的分布:
对每个 $y$ 求集合 $A_y = \{x : r(x) \leq y\}$;
求 CDF:$F_{r(X)}(y) = \Pr{r(X) \leq y} = \Pr{X \in A_y} = \int_{A_y} f_X(x) \d x$;求 PDF:$f_Y(y) =  F_Y'(y)$。当 $r$ 单调时,$r$ 的反函数为 $s = r^{-1}$,有 $f_Y(y) = f_X(s(y)) \left|\frac{\d s(y)}{\d y}\right|$。\\
例:$f_X(x) = e ^ {-x} (x > 0)$, $F_X(x) = \int_{0}^{x} f_X(s)\d s = 1 - e ^ {-x}$。令 $Y = r(X) = \log X, A_y = \{x : x \leq e ^ y\}$,$F_Y(y) = \Pr{X \leq e ^ y} = F_X(e ^ y) = 1 - e ^ {-e ^ y}$, $f_Y(y) = F_Y'(y) = e ^ y e ^ {-e ^ y}$。\\	
作业:令 $X, Y \sim \mathrm{Uniform} (0, 1)$ 且独立,求 $X - Y$ 的 PDF。令 $Z = X - Y$,
$ F_{Z} (z) = \int_{x + y \leq z} f(x, y) \d x \d y =
\begin{cases}
0 & \text{if $z \leq -1$} \\
\frac{(1 + a)^2}{2} & \text{if $-1 < z < 0$} \\
1 - \frac{(1 - a)^2}{2} & \text{if $0 \leq z < 1$} \\
1 & \text{if $z \geq 1$}
\end{cases}$ $\begin{cases}
f_Z(z) = \\
1 + z, \quad \text{$-1 < z < 0$} \\
1 - z, \quad \text{$0 \leq z < 1$} \\
0, \quad \text{otherwise}
\end{cases}
$
		\newcolumn
		
		test 

		\newcolumn
		
		test
	\end{multicols}
	\newpage
	test
	
\end{document}
%THE SCL ENDS
